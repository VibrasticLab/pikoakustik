\documentclass[12pt,]{article}
\usepackage[utf8]{inputenc}
\usepackage[T1]{fontenc}
\usepackage{mathptmx}
\usepackage{geometry}
\usepackage{mathtools}
\usepackage[english]{babel}
\usepackage{graphicx}
\usepackage{stackengine}
\usepackage[os=win]{menukeys}
\usepackage{hyperref}
\usepackage{minted}
\usepackage{xcolor}
\usepackage{tikz}
\usepackage[yyyymmdd,hhmmss]{datetime}
\usepackage{etoolbox}

\patchcmd{\thebibliography}{\section*{\refname}}{}{}{}

\newcommand{\ShowOsVersion}{
	\immediate\write18{\unexpanded{foo=`uname -sro` && echo "${foo}" > tmp.tex}}
	\input{tmp}\immediate\write18{rm tmp.tex}
}

\newcommand{\ShowTexVersion}{
	\immediate\write18{\unexpanded{foo=`pdflatex -version | head -n1 | cut -d' ' -f1,2` && echo "${foo}" > tmp.tex}}
	\input{tmp}\immediate\write18{rm tmp.tex}
}

\addto\captionsenglish{\renewcommand{\contentsname}{Daftar Isi}}

\hypersetup{
	colorlinks=true, %set true if you want colored links
	linktoc=all,     %set to all if you want both sections and subsections linked
	linkcolor=blue,  %choose some color if you want links to stand out
}

\geometry{
	legalpaper,
	left=15mm,
	right=10mm,
	top=10mm,
	bottom=15mm,
}

\title{\Large \bf
	License and Patent Claim Draft\\
	\small{(Mid-Testing Stage)}
}

\author{Achmadi ST MT}

\date{}

\hypersetup{citecolor=black}

%\pagecolor[rgb]{0.1,0.1,0.1}
%\color[rgb]{1,1,1}

\begin{document}
	\maketitle
	\thispagestyle{empty}
	
	\vspace*{600pt}
	\noindent Seluruh konten dalam dokumen ini mengacu kepada pengembangan unit Audiometri di \textit{commit} terakhir:\\
	\url{https://github.com/VibrasticLab/pikoakustik/commit/fa1db9a040644eab3fd033c1f8c43e28af1f0ebf}.\\
	
	\noindent This report written using: \\
	OS : \ShowOsVersion \\
	TeX : \ShowTexVersion \\
	Update: {\today} at \currenttime \\
	
	%%%%%%%%%%%%%%%%%%%%%%%%%%%%%%%%%%%%%%%%%%%%%%%%%%%%%%%%%%%%%%%%%
	
	\newpage
	\tableofcontents
	
	%%%%%%%%%%%%%%%%%%%%%%%%%%%%%%%%%%%%%%%%%%%%%%%%%%%%%%%%%%%%%%%%%
	
	\newpage
	\section{Software/Firmware}
	
	Berikut akan dijabarkan beberapa aspek dari sisi software/firmware yang dijalankan di hardware.
	Seluruh klaim pada aspek ini dalam bentuk klaim kode sumber (\textit{source-code}) dan tidak terbatas hanya \textit{binary} akhir.
	
	\subsection{Libraries}
	Disini akan dijabarkan pengunaan pustaka (libraries) yang digunakan dalam pengembangan.
	Seluruh pustaka disini berupa kode sumber implementasi dan API (\textit{Application Programming Interface})
	di luar implementasi dan perbaikan (\textit{patch}) yang dilakukan oleh pihak peneliti disini.
	
	Penjabaran ini perlu mengingat pustaka tersebut dikembangankan oleh pihak lain dan telah memiliki klaim lisensi sendiri.
	Sehingga secara otomatis pihak peneliti disini \textbf{tidak bisa mengklaim} pustaka tersebut.
	
	\begin{itemize}
		\item \textbf{GNU GCC ARM}. Merupakan sekumpulan \textit{toolchain} dan pustaka untuk kompilasi source-code ke binary untuk chip ARM.
		Dikembangan oleh ARM Limited dengan lisensi yang digunakan adalah GPL versi 3.0\\
		\url{https://developer.arm.com/open-source/gnu-toolchain/gnu-rm}
		
		\item \textbf{ChibiOS/RT}. Merupakan sekumpulan pustaka yang berisi implementasi dan API abstraksi untuk chip ARM seri Cortex-M.
		Dikembangkan oleh Giovanni Di Sirio dengan lisensi yang digunakan adalah GPL versi 3.0 dan Apache versi 2.0\\
		\url{https://www.chibios.org/dokuwiki/doku.php}
		
		\item \textbf{FatFS}. Merupakan sekumpulan pustaka yang berisi implementasi dan API abstraksi untuk FAT16/FAT32 \textit{filesystem}.
		Pustaka ini digunakan untuk menangani berkas-berkas teks yang disimpan di \textit{memory-card}.
		Dikembangkan oleh Elm ChanN dengan lisensi yang digunakan adalah BSD license.\\
		\url{http://elm-chan.org/fsw/ff/00index_e.html}
		
		\item \textbf{ESP-Open-SDK}. Merupakan sekumpulan \textit{toolchain} dan pustaka untuk kompilasi source-code ke binary untuk platform ESP8266/EX.
		Dikembangan oleh Tensilica Inc. dan Espressif Inc. dengan lisensi yang digunakan adalah GPL versi 2.0\\
		\url{https://github.com/pfalcon/esp-open-sdk}
		
		\item \textbf{esp\_mqtt}. Merupakan sekumpulan pustaka yang berisi implementasi protocol MQTT Client untuk ESP8266/EX.
		Dikembangkan oleh Tuan PM dengan lisensi yang digunakan adalah MIT License.\\
		\url{https://github.com/tuanpmt/esp_mqtt}
	\end{itemize}

	\subsection{Audio}
	
	Berikut dijabarkan beberapa klaim terhadap metode/algoritma terkait fitur Audio/Tone.
	
	\subsubsection{Tone Generation}
	
	Referensi persamaan untuk \textit{tone generation}, tersedia \textbf{ht\_audio.c}.
	Aspek klaim adalah pola persamaan dan seluruh definisi konstanta yang digunakan,
	terkecuali objek struktur konfigurasi I2S.
	
\begin{minted}[frame=lines,framesep=2mm,fontsize=\small]{c}
#define USE_STEREO_ARRAY TRUE

#define I2S_BUFF_SIZE 512
#define DEFAULT_ATTEN 0.01
#define DEFAULT_AMPL_THD 1

#define TOTAL_BUFF_SIZE I2S_BUFF_SIZE*16

void ht_audio_Zero(void){
	uint16_t i;
	for(i=0;i<TOTAL_BUFF_SIZE;i++){
		i2s_tx_buf[i] = 0;
	}
}

void ht_audio_Tone(double freq, double ampl){
	uint16_t i;
	uint16_t buffsize;
	double ysin;
	double ampl_act;
	
	buffsize = (uint16_t) I2S_BUFF_SIZE/freq;
	
	ampl_act = DEFAULT_ATTEN*ampl*32767;
	if(ampl_act<=DEFAULT_AMPL_THD){ampl = 0;}
	
	ht_audio_Zero();
	
	for(i=0;i<buffsize;i++){
		ysin = DEFAULT_ATTEN*ampl*32767*sin(2*3.141592653589793*((double)i/(double)buffsize));
	
		if(ysin >= 0){
			i2s_tx_buf[i]=ysin;
#if USE_STEREO_ARRAY
			i2s_tx_buf[i+1]=ysin;
#endif
	}
		if(ysin < 0){
			i2s_tx_buf[i]=ysin+65535;
#if USE_STEREO_ARRAY
			i2s_tx_buf[i+1]=ysin+65535;
#endif
		}
	}
	
	i2scfg.size = buffsize; //Can't claimed
}
\end{minted}

	\subsubsection{Tone Play}
	
	Referensi metode untuk \textit{tone playing}, tersedia \textbf{ht\_audio.c}.
	Aspek klaim adalah durasi dan flow metode, terkecuali objek struktur konfigurasi I2S.

\begin{minted}[frame=lines,framesep=2mm,fontsize=\small]{c}
void ht_audio_Play(uint16_t duration){
	i2sStart(&I2SD2, &i2scfg);
	i2sStartExchange(&I2SD2);
	
	chThdSleepMilliseconds(duration*10);
	
	i2sStopExchange(&I2SD2);
	i2sStop(&I2SD2);
}
\end{minted}

	\subsubsection{L/R Control}
	
	Referensi metode untuk kendali channel kiri-kanan, tersedia \textbf{ht\_audio.c}.
	Aspek klaim adalah flow metode, terkecuali nomor pin pada chip STM32.
	
	\begin{minted}[frame=lines,framesep=2mm,fontsize=\small]{c}
void ht_audio_Init(void){
	palSetPadMode(AUDIO_IO,AUDIO_L,PAL_MODE_OUTPUT_PUSHPULL);
	palSetPadMode(AUDIO_IO,AUDIO_R,PAL_MODE_OUTPUT_PUSHPULL);
}
	
void ht_audio_DisableCh(void){
	palClearPad(AUDIO_IO,AUDIO_L);
	palClearPad(AUDIO_IO,AUDIO_R);
}

void ht_audio_LeftCh(void){
	ht_audio_DisableCh();
	palSetPad(AUDIO_IO,AUDIO_L);
}

void ht_audio_RightCh(void){
	ht_audio_DisableCh();
	palSetPad(AUDIO_IO,AUDIO_R);
}
	\end{minted}

	\newpage
	\section{Hardware}
\end{document}